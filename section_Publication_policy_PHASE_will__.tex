\section{Publication policy}
PHASE will ensure that the models submitted to the repository (see section \ref{sec:repo}) are attributed to the original authors. Existing publications will be used as citations where they are available. In cases when a model is created collaboratively on the repository, alternative methods of attribution, such as \href{http://openbadges.org}{software badges} will be provided. When a collaborative model reaches a state of sophistication that allows it to be used in an analysis, PHASE will assist in the publication of a paper with an author list containing all contributors to the code.

In those cases where a method developed in the PHASE framework or code from the PHASE repository are used in a data analysis, the experimental collaboration owning the data set commits to offer the developers of that method an associated membership for the duration of that data analysis project. The associated membership implies rights and obligations. It provides the right to authorship on the publication(s) being produced from the specific analysis effort. It includes the obligation, detailed by the experimental collaboration, to report to the collaboration organs and submit to the internal reviewing procedures before publicly presenting the obtained results. 

Associated memberships are always bilateral agreements between guest researchers and experiments. PHASE member collaborations commit to creating such a mechanism. The PHASE panel acts as a facilitator to bring together external experts and experimentalists.