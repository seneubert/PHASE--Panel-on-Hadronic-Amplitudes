\section{Activities}
\label{sec:activities}

\subsection{PHASE Repository}
\label{sec:repo}
PHASE will curate a repository of amplitude models. The models will consist of computer codes with executable examples and documentation. Care will be taken to ensure proper attribution of the contributed code and extensive documentation of the properties, the applicability and the limitations of the available models. The repository web interface will provide the infrastructure for discussion, review and development of the model code. An example for such an interface is \href{https://about.gitlab.com/}{gitlab}. 

Already many software packages for amplitude analysis are being developed and maintained across the experimental community. PHASE considers the diversity of this software ecosystem as a valuable asset. Especially for the implementation of high performance fitters, which can efficiently calibrate a model against a dataset, several techniques and approaches are being currently investigated, ranging from intelligent caching of intermediate results to different parallel computing architectures. Therefore the PHASE repository is not meant to provide a framework for amplitude analysis, which would replace existing solutions. It focuses on the underlying physics implemented in the models and delegates performance improvements to the experiment specific tools. The code collected in the repository should be seen as a reference library against which other implementations can be validated (see subsection \ref{sec:validation}). 

PHASE will collect and provide links to all the publicly available fitters used by the partaking experiments. The PHASE repository can also serve as a long time archive to secure copies of analysis code which no longer is maintained by a specific experiment.

The PHASE repository aims at providing an active and lively forum for exchange between analysts, software developers and theorists. To facilitate this, the repository will be completely open for contributions. All contributed code will be distributed under the Creative Commons
\href{https://creativecommons.org/licenses/by-sa/4.0/}{Attribution-ShareAlike 4.0 International} license. The members of the panel will assume the roles of moderators on the discussion and review portion of the repository website. All participants agree to adhere to the \href{https://rfc.zeromq.org/spec:42/C4/}{ZeroMQ collective code construction contract C4}, which will be updated and adapted to the need of the community by the PHASE panel.

PHASE will provide the means to attribute the models submitted to the repository to the original authors. Original publications will be used where they are available. In cases when a model is created collaboratively on the repository, alternative methods of attribution, such as \href{http://openbadges.org}{software badges} will be provided. When a collaborative model reaches a state of sophistication that allows it to be used in an analysis, PHASE will assist in the publication of a paper with an author list containing all contributors to the code.

\subsection{Hadron amplitude hackathons}
\label{sec:hackathon}
PHASE will organize regular workshops, which will take the form of hackathons. Experts work on a concrete problem, such as providing a certain parameterization to the analysts. Each hackathon will have a specific goal and will deliver actual software. The resulting tools are provided to the experiments through the PHASE repository. The attribution of the developed models follows the guidelines lined out in subsection \ref{sec:repo}.

In order to test and develop these tools the experimental collaborations will be asked to provide simulated data, or, where applicable, recorded data sets. 

\subsection{Annual review on amplitude analysis}
\label{rec:review}
The members of the panel compile an annual review article to be published in a major journal of their choice. The aim of this article is to review the state of the art of hadronic amplitude models and highlight best practices. Key results and outstanding challenges in hadronic spectroscopy are reviewed. The panel ensures that the models and techniques discussed in the PHASE annual review are made available in the repository. It encourages an open discussion of the reviewed items in the repository forum.

\subsection{Amplitude model validation}
\label{sec:validation}
PHASE will provide the expertise to validate the models used by individual analyses by the partaking experiments. This validation is a community service, which can be tapped into by the collaborations as a resource to inform the internal review of ongoing analyses. Validation can take two forms: the experiments can directly use the source code provided and reviewed in the PHASE repository for their analyses. Or they can proof, using toy Monte Carlo samples, that the models used in a particular analysis give identical results (or possibly deviate) from the PHASE reference implementations.

\subsection{Community building}
\label{sec:community}
An important goal of PHASE is to intensify the exchange of ideas between the participants. In addition to hosting the model repository and organizing the amplitude hackathons, PHASE will facilitate community building measures. Experiment and theory liaisons will identify common ground and potential synergies across experiments and theory groups. They will establish contacts between students working on related projects and engage them in knowledge transfer.

Distributed teamwork is a considerable challenge and needs modern communication tools to succeed. PHASE will operate a virtual team workplace, such as a \href{http://slack.com}{slack} community to provide an address for live questions and exchange on all issues surrounding amplitude analysis and hadron physics. This virtual community targets in particular students who are engaged in the day-to-day business of data analysis and programming.

\subsection{Open data advocacy}
\label{sec:opendata}
PHASE recognizes the authority of all experimental collaborations over their data. There is however, increasing pressure from the public and from  funding agencies to make results and rw data publicly available. The complex nature of data and models used in amplitude analyses, open data formats, which allow the meaningfull interpretation of the data b In order to further the usage of data across the community the network will support collaborations when they decide to publish a dataset, usually after the initial analysis of the data has been published by the primary authors of the data. 

PHASE provides a platform for experiments to develop the technology and negotiate the terms under which data is publicly shared. The network advocates open data policies and will search methods to ensure proper technical treatment of the data as well as proper attribution.


\subsection{PHASE fellowships}
\label{sec:fellowships}
PHASE will seek to provide four 2-year fellowships every two years. Two will have theoretical and two of theses positions will have experimental/computational focus. The fellowships will be awarded to young researchers with expertise in amplitude analysis. The program is open internationally and will provide a means to attract talented physicists to the challenges of hadron physics. Fellows will join the core team, taking up responsibilities for the technical tasks and will work on specific physics questions within the PHASE network. PHASE will strive to co-locate its fellows. Travel funds will be provided taking into account the particular role of network multipliers the fellows will play. All PHASE fellows will become members of the PHASE annual review author list for the duration of their fellowship plus the following year.     
