\section{Activities}
\label{sec:activities}
\subsection{PHASE Repository}
PHASE will curate a repository of amplitude models. The models will consist of computer codes with executable examples and documentation. Care will be taken to ensure proper attribution of the contributed code and extensive documentation of the properties and in particular the limitations of the various models. The repository web interface will provide the infrastructure for discussion, review and development of the model code. An example for such an interface is \href{https://about.gitlab.com/}{gitlab}. 

The PHASE repository aims at providing an active and lively forum for exchange between analysts, software developers and theorists. To facilitate this, the repository will be completely open for contributions. All contributed code will be distributed under the Creative Commons
\href{https://creativecommons.org/licenses/by-sa/4.0/}{Attribution-ShareAlike 4.0 International} license. The members of the panel will assume the roles of moderators on the discussion and review portion of the repository website. All participants agree to adhere to the \href{https://rfc.zeromq.org/spec:42/C4/}{ZeroMQ collective code construction contract C4}, which will be updated and adapted to the need of the community by the PHASE panel.

\subsection{Hadron Amplitude Hackathons}
PHASE will organize regular workshops, which will take the form of Hackathons where experts work on a concrete problem, such as providing a certain parameterization to the analysts. Each hackathon will have a concrete goal and will deliver actual software, which is provided to the experiments through the PHASE repository.

In order to test and develop these tools the experimental collaborations will be asked to provide simulated data, or where applicable, measured data sets. 

\subsection{Annual review on amplitude analysis}

\subsection{Amplitude model validation}

\subsection{Community building}

\subsection{Open data advocacy}

\subsection{PHASE fellowships}