\subsection{Liaisons}
Each collaboration sends two liaisons as representatives into the PHASE panel. The experiment liaisons represent their collaboration on the panel. Each partaking theory group can send one liaison into the panel. Liaisons inform their collaborators on the PHASE activities and advocate for the open exchange of models and code. They commit to the following responsibilities:
\begin{itemize}
\item To ensure the flow of information between the partaking institutions; they commit to attending PHASE panel meetings and to give regular reports of PHASE activities at their collaboration meetings
\item Curate the PHASE repository; panel members act as moderators of the repository discussion forum and commit to respond in a timely manner to forum requests on their area of expertise. 
\item Panel members uphold a professional standard of neutrality during the curation. They ensure open access to the repository and commit to facilitating a fair scrutiny of the contributed models by the community. 
\item They commit to contribute to the publication of a regular review article on the state of art of amplitude analyses.
\item They draft joint funding applications for the PHASE project.
\item Panel member decide on allocation of jointly acquired funding.
\end{itemize}

\subsection{Core Team}
In addition to the PHASE panel we install a core team running the day-to-day business of the PHASE infrastructure. This core team will consist of volunteers (which can be panel members) and at a later state is foreseen to employ part-time or full-time positions created through joint funding. It will operate the more technical aspects of the PHASE repository and help contributors with technical questions. The core team will also be responsible for the technical organization of PHASE hackathons and further activities listed in in section \ref{sec:activities}.

