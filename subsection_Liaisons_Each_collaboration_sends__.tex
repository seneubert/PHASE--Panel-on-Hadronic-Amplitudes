\subsection{Liaisons}
Each collaboration sends two liaisons as representatives into the PHASE panel. The experiment liaisons represent their collaboration on the panel. Each partaking theory group can send one liaison into the panel. Liaisons inform their collaborators on the PHASE activities and advocate for the open exchange of models and code. They commit to the following responsibilities:
\begin{itemize}
\item To ensure the flow of information between the partaking institutions; they commit to regular reports of PHASE activities at their collaboration meetings
\item Curate the PHASE repository; panel members act as moderators of the repository discussion forum and commit to respond within 48h to forum requests on their area of expertise. 
\item Panel members uphold a professional standard of neutrality during the curation. They ensure open access to the repository and ommmit to a fair scrutiny of the contributed models. 
\item They commit to contribute to the publication of a regular review article on the state of art of amplitude analyses.
\item Draft joint funding applications for the PHASE project;
\item Decide on allocation of jointly acquired funding;
\end{itemize}

\subsection{Core Team}
In addition to the PHASE panel we foresee a core team which runs the day-to-day business of the PHASE infrastructure. This core team will consist of volunteers (which can be panel members) and at a later state is foreseen to employ part-time or full-time positions created though joint funding. The core team will operate the more technical aspects of the PHASE repository and help contributors with technical questions. It will also be responsible for the organization of PHASE Hackathons and further activities listed in in section \ref{sec:activities}.

\subsection{Founding institutions and partners}
PHASE was founded by members of the following collaborations and groups.

Experimental Collaborations: Belle II, BES III, COMPASS, LHCb \\
Theory groups: Bonn, J\"ulich, Siegen, Madrid

PHASE seeks to cooperate with the Joint Physics Analysis Center \href{https://jpac.jlab.org/}{JPAC}, which pursues similar goals in the US.

The initiative is open to any interested experiment and theory groups.