\section{Vision and Goals}

The vision of PHASE is to enable joint analyses of hadronic processes across multiple experiments and multiple data sets. It provides a forum and infrastructure for joint developments within the hadron physics community and defines best practices.

To make progress in the understanding of strong interaction phenomena, it is necessary to combine the information gathered by several experiments across a wide range of energies in a coherent fashion. Due to the complexity of hadron dynamics, this is a non-trivial problem. Current state-of-the-art techniques for extracting hadronic amplitudes from data exhibit model dependencies, which need to be quantified. Only then will the envisioned global analyses become possible. One of the main reasons for the difficulties currently experienced is that in hadronic decays often many processes are coupled together. For a comprehensive treatment all of them need to be taken into account, while each analysis typically only investigates a single final state. It is at this point that combining data from different experiments will have the biggest impact, as in this scenario each channel can be constrained from the experimental data which is most sensitive to that particular hadronic system.

To advance further, methods are being developed which will enable the extraction of universal features from the data. In addition, procedures for data handling across experiments are needed, which will allow the transfer of those universal features between different production processes and final states. Theorists have already developed the basic concepts how such universal features can be defined. However, considerable technical challenges have so far prohibited the full implementation of these ideas into the data analysis frameworks of the current generation of experiments. The outstanding challenges and possible angles of attack have been discussed and documented in a series of workshops during the past few years \cite{Battaglieri_2015, Brice_o_2016, Lutz_2016}. It is now a consensus in the community that to overcome said challenges, a closer cooperation between phenomenologists and experimentalists is needed.

The first and foremost goal of the PHASE network is to facilitate this cooperation. 

The activities carried out in the project proposed here aim at providing researchers with useful tools for data analysis. They will produce actual software implementations of amplitude models. The focus will be on advanced models that cannot be produced by a single collaboration but need combined expertise from several actors. PHASE puts the software implementation of the physics models into the center of its activities. Its goal is to advance complex analysis projects from the conceptual to the actual.





