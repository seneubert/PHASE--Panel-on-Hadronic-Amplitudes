\section{Vision and Goals}

The vision of PHASE is to enable joint analysis of hadronic processes across multiple experiments and multiple data sets. It provides a forum and infrastructure for joint developments within the hadron physics community and defines best practices.

To make progress in the understanding of strong interaction phenomena, it is necessary to combine the information gathered by several experiments across a wide range of energies in a coherent fashion. Due to the complexity of hadron dynamics, this is a non-trivial problem. Current state-of-the-art techniques for extracting hadronic amplitudes from data exhibit model dependencies, which inhibit such global analyses. The associated problems also make it difficult to use high precision measurements, obtained in a dedicated experiment, as constraints for analyses of other data sets. 

Progress can be made, if first methods are developed to extract universal features from the data. Second, procedures for data handling have to be developed across experiments, which allow the transfer of those universal features. Theorists have developed the basic ideas on how such universal features could be defined. However, considerable technical challenges have so far prohibited the full implementation of theses ideas into the data analysis frameworks of the current generation of experiments. The outstanding challenges and possible angles of attack have been discussed and documented in a series of workshops during the past few years \cite{Battaglieri_2015, Brice_o_2016, Lutz_2016}. It is now a consensus in the community that to overcome said challenges, a closer exchange between phenomenologists and experimentalists is needed.

The first and foremost goal of the PHASE network is to facilitate this exchange. 




