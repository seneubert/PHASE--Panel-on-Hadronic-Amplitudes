\section{PHASE Structure}
\label{sec:structure}
PHASE consists of two representatives (liaisons) from each experiment and a number of representatives (liaisons) from the theory community. Experimental plus theory liaisons make up the PHASE panel. They elect a speaker to represent PHASE to the public and funding agencies. The panel holds regular virtual meetings to coordinate the activities outlined in section \ref{sec:activities}. The the panel assumes the following responsibilities:
\begin{itemize}
\item ensure the flow of information between the partaking institutions
\item curate the PHASE repository
\item ensure the publication of an annual review article on the state of art of amplitude analyses
\item draft joint funding applications for the PHASE project
\item decide on allocation of jointly acquired funding
\end{itemize}



\subsection{Experiment Liaisons}

\subsection{Theory Liaisons}

\subsection{Core Team}

In addition to the PHASE panel we foresee a core team which runs the day-to-day business of the PHASE infrastructure. This core team will consist of volunteers (which can be panel members) and at a later state is foreseen to employ part-time or full-time positions created though joint funding. The core team will operate the more technical aspects of the PHASE repository and help contributors with technical questions. It will also be responsible for the organization of PHASE Hackathons and further activities listed in in section \ref{sec:activities}.

\subsection{Founding institutions}
Experimental Collaborations: Belle II, BES III, COMPASS, LHCb 
Theory groups: Bonn, J\"ulich, Siegen, Valencia

The initiative is open to any interested experiment and theory group.
